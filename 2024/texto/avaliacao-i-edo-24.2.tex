\documentclass[12pt,a4paper]{article}
\usepackage[utf8]{inputenc}
\usepackage[brazil]{babel}
\usepackage{graphicx}
\usepackage{amssymb, amsfonts, amsmath}
\usepackage{float}
\usepackage{enumerate}
\usepackage[top=2.5cm, bottom=2.5cm, left=1.25cm, right=1.25cm]{geometry}

\DeclareMathOperator{\sen}{sen}

\begin{document}
\pagestyle{empty}

\begin{center}
  \begin{tabular}{ccc}
    \begin{tabular}{c}
      \includegraphics[scale=0.25]{../../biblioteca/imagem/brasao-de-armas-brasil} \\
    \end{tabular} & 
    \begin{tabular}{c}
      Ministério da Educação\\
      Universidade Federal dos Vales do Jequitinhonha e Mucuri\\
      Faculdade de Ciências Sociais, Aplicadas e Exatas - FACSAE\\
      Departamento de Ciências Exatas - DCEX\\
      Disciplina: Introdução às Equações Diferenciais Ordinárias\\
      Semestre: 2024/2\\
      Prof. Dr. Luiz C. M. de Aquino\\
      Aluno(a):\rule{6cm}{0.1mm} \quad Data: \rule{0.5cm}{0.1mm}/\rule{0.5cm}{0.1mm}/\rule{1cm}{0.1mm}\\
    \end{tabular} &
    \begin{tabular}{c}
      \includegraphics[scale=0.25]{../../biblioteca/imagem/logo-ufvjm} \\
    \end{tabular}
  \end{tabular}
\end{center}

\begin{center}
 \textbf{Avaliação I}
\end{center}

\textbf{Instruções}
\begin{itemize}
 \item Todas as justificativas necessárias na solução de cada questão devem 
 estar presentes nesta avaliação;
 \item Esta avaliação tem um total de 30,0 pontos.
\end{itemize}

\begin{enumerate}
  \item \textbf{[10 pontos]} Resolva as equações abaixo pelo método da separação de variáveis.

  \begin{enumerate}
    \item $\dfrac{1}{4}y' = \dfrac{x^3}{y}$
    \item $\dfrac{dy}{dx} = \dfrac{x - e^{-3x}}{y + e^{2y}}$
  \end{enumerate}

  \item \textbf{[10 pontos]} Resolva as equações abaixo usando um fator de integração.

  \begin{enumerate}
    \item $ty' + y = \sen t$
    \item $ty'+ 3y = t^2 - t$
  \end{enumerate}

  \item \textbf{[10 pontos]} Resolva as equações exatas abaixo.

  \begin{enumerate}
    \item $\left(xy^2 + 3x^2y\right)dx + \left(x + y\right)x^2 dy = 0$
    \item $(x\ln y + xy)dx + \dfrac{x^2(1 + y)}{2y}dy = 0$
  \end{enumerate}
  
\end{enumerate}

\end{document}