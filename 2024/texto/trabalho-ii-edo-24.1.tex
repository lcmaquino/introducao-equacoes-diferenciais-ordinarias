\documentclass[12pt,a4paper]{article}
\usepackage[utf8]{inputenc}
\usepackage[brazil]{babel}
\usepackage{graphicx}
\usepackage{amssymb, amsfonts, amsmath}
\usepackage{float}
\usepackage{enumerate}
\usepackage[top=2.5cm, bottom=2.5cm, left=1.25cm, right=1.25cm]{geometry}

\DeclareMathOperator{\sen}{sen}
\DeclareMathOperator{\tg}{tg}

\begin{document}
\pagestyle{empty}

\begin{center}
  \begin{tabular}{ccc}
    \begin{tabular}{c}
      \includegraphics[scale=0.25]{../../biblioteca/imagem/brasao-de-armas-brasil} \\
    \end{tabular} & 
    \begin{tabular}{c}
      Ministério da Educação \\
      Universidade Federal dos Vales do Jequitinhonha e Mucuri \\
      Faculdade de Ciências Sociais, Aplicadas e Exatas - FACSAE \\
      Departamento de Ciências Exatas - DCEX \\
      Disciplina: Introdução às Equações Diferenciais Ordinárias \\
      Semestre: 2024/1 \\
      Prof. Dr. Luiz C. M. de Aquino \\
    \end{tabular} &
    \begin{tabular}{c}
      \includegraphics[scale=0.25]{../../biblioteca/imagem/logo-ufvjm} \\
    \end{tabular}
  \end{tabular}
\end{center}

\begin{center}
  \textbf{Trabalho II}
\end{center}

\begin{enumerate}
  \item Resolva os sistemas de equações abaixo com suas respectivas condições inicias.

  \begin{enumerate}
    \item 
    $\begin{cases}
      \dfrac{dx}{dt} = 3x - 2y \\ \\
      \dfrac{dy}{dt} = 2x - 2y \\
    \end{cases},\quad x(0)=3,\, y(0)=\dfrac{1}{2}$
    
    \item 
    $\begin{cases}
      \dfrac{dx}{dt} = x - 2y \\ \\
      \dfrac{dy}{dt} = 3x - 4y \\
    \end{cases},\quad x(0)=-1,\, y(0)=2$
  
  \end{enumerate}

\end{enumerate}

\end{document}